\documentclass{../cssheet}

%--------------------------------------------------------------------------------------------------------------
% Basic meta data
%--------------------------------------------------------------------------------------------------------------

\title{Aufgaben zum Aufwärmen -- Teil 2}
\author{Prof. Dr. Christian Spannagel}
\date{\today}
\hypersetup{%
    pdfauthor={\theauthor},%
    pdftitle={\thetitle},%
    pdfsubject={Aufgabenblatt Geometrie},%
    pdfkeywords={geometrie}
}


%--------------------------------------------------------------------------------------------------------------
% document
%--------------------------------------------------------------------------------------------------------------

\begin{document}
\printtitle

\textbf{Aufgabe 4 (Platonische Körper):}  Platonische Körper (oder auch reguläre Körper) bestehen aus kongruenten regelmäßigen Vielecken. An jeder Ecke stoßen gleich viele Vielecke aufeinander. Insgesamt gibt es nur $5$ Platonische Körper.

\begin{enumerate}[(a)]
\item Welches sind die platonischen Körper? Versucht zunächst, so viele wie möglich durch Überlegung zu finden. Sucht anschließend im Netz nach den restlichen Körpern.
\item Warum gibt es nur diese 5 platonischen Körper? Findet eine schlüssige Begründung!
\item Wie viele Flächen, Ecken und Kanten haben jeweils die platonischen Körper? Löst diese Aufgabe ohne Hilfsmittel. Hier geht es darum, dass ihr eure Vorstellungskraft schult.
\item In allen konvexen Polyedern -- nicht nur bei den platonischen Körpern -- besteht eine besondere Beziehung zwischen der Anzahl der Ecken, der Kanten und der Flächen. Entdeckt ihr den Zusammenhang? Beweist ihn! (Tipp: Projiziert das Polyeder auf die Ebene derart, dass die Kanten sich nur in den Ecken schneiden. Geht das immer? Das, was dabei herauskommt, nennt man einen zusammenhängenden planaren Graphen. Beweist eure Vermutung dann mit vollständiger Induktion für alle zusammenhängenden planaren Graphen.)
\end{enumerate}

\textbf{Aufgabe 5 (Satz des Pythagoras):} Beweist den Satz des Pythagoras ikonisch (d.\,h. bildhaft). Findet verschiedene Beweise!


%\pagestyle{docstyle}
\vspace*{10mm}

\printlicense

\printsocials


\end{document}
