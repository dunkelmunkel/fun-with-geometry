\documentclass{../cssheet}

%--------------------------------------------------------------------------------------------------------------
% Basic meta data
%--------------------------------------------------------------------------------------------------------------

\title{Aufgaben zum Aufwärmen}
\author{Prof. Dr. Christian Spannagel}
\date{\today}
\hypersetup{%
    pdfauthor={\theauthor},%
    pdftitle={\thetitle},%
    pdfsubject={Aufgabenblatt Geometrie},%
    pdfkeywords={geometrie}
}

%--------------------------------------------------------------------------------------------------------------
% document
%--------------------------------------------------------------------------------------------------------------

\begin{document}
\printtitle

\textbf{Vorbemerkung:}  Die Aufgaben dienen unter anderem dazu, dass ihr euch bewusst macht, welche Techniken und Strategien ihr beim Lösen solcher Aufgaben einsetzen könnt. Macht euch eine Liste aller Strategien! Wir diskutieren anschließend, welche Strategien ihr eingesetzt habt und welche zum Erfolg beigetragen haben!

\textbf{Aufgabe 1 (Ecken und Kanten):}  Wie viele Seiten und wie viele Diagonalen hat ein $n$-Eck? Und in der Summe? Findet die Muster, packt sie in Formeln, und beweist diese!

\textbf{Aufgabe 2 (Die inneren Werte):} Wie groß ist die Summe der Innenwinkel eines $n$-Ecks? Findet das Muster, packt es in eine Formel, und beweist diese auf mindestens zwei verschiedene Arten.

\textbf{Aufgabe 3 (Parkettboden):} Mit welchen regelmäßigen $n$-Ecken kann man die Ebene lückenlos parkettieren? Findet einen Beweis.


\vspace*{10mm}

\printlicense

\printsocials



\end{document}
