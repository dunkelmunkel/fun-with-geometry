\documentclass{../cssheet}

%--------------------------------------------------------------------------------------------------------------
% Basic meta data
%--------------------------------------------------------------------------------------------------------------

\title{Fun with Instagram!}
\author{Prof. Dr. Christian Spannagel}
\date{\today}
\hypersetup{%
    pdfauthor={\theauthor},%
    pdftitle={\thetitle},%
    pdfsubject={Aufgabenblatt Geometrie},%
    pdfkeywords={geometrie}
}


%--------------------------------------------------------------------------------------------------------------
% document
%--------------------------------------------------------------------------------------------------------------

\begin{document}
\printtitle

Lasst uns in diesem Semester gemeinsam erfahren, welche Effekte entstehen, wenn man auf Social Media mathematischen Content postet!

\begin{enumerate}
\item \textbf{Gruppenbildung:} Bildet eine Gruppe von 3 bis 4 Personen. Jede Gruppe sollte mindestens ein Instagram-Profil verwenden, das entweder bereits existiert oder neu erstellt wird. Es ist ausreichend, wenn eine Person aus der Gruppe das Profil verwendet. Gerne könnt ihr aber auch mehr als ein Profil bedienen.
\item \textbf{Profilgestaltung:}  Gestaltet das Profil ansprechend und professionell, insbesondere wenn es neu erstellt wurde. Achtet dabei auf ein passendes Profilbild, eine aussagekräftige Headline und eine kurze, aber informative Zusammenfassung der Interessen und Ziele. Folgt damit außerdem interessanten Accounts, z.B. von Lehrer*innen, denn eure eigenen Inhalte werden nur wahrgenommen, wenn ihr selbst vernetzt seid.
\item \textbf{Erstellung des ersten Posts:} Sucht euch ein Motiv aus dem Alltag, das einen besonderen Bezug zu Geometrie besitzt. Welche besondere geometrische Form habt ihr entdeckt? An welchen Gegenständen eurer Umgebung kann man gut geometrische Sachverhalte erkennen? Erstellt daraus einen Post, entweder ein Bild oder ein kurzes Reel. Erstellt dazu eine kurze Beschreibung, in der ihr eure Gedanken dazu aufschreibt. Am besten, ihr stellt dort (auch) eine Frage an eure Follower*innen, denn dadurch erhöht ihr das Engagement für euren Post. Ergänzt relevante Hashtags (\#funwithgeomtry \#phheidelberg \#meinephhd \#instalehrerzimmer \ldots) und verlinkt, wenn möglich, @phheidelberg oder andere relevante Profile, um die Sichtbarkeit eures Posts zu erhöhen. Verlinkt auch eure Profile gegenseitig!
\item \textbf{Dokumentation im Moodle-Forum:} Setzt den Link zum Post direkt nach dem Posten ins Moodle-Forum. Dokumentiert dabei den Prozess und das Ergebnis eurer Arbeit kurz im Moodle-Forum. Beantwortet dabei folgende Fragen: Welches Motiv habt ihr gewählt und warum? Was waren eure Überlegungen bei der Erstellung des Posts?
\item \textbf{Interaktion:} Kommentiert mindestens zwei Posts anderer Gruppen. 
\end{enumerate}

Hinweis: Bitte beachtet unbedingt das Urheberrecht und verletzt keine Persönlichkeitsrechte!


\vspace*{10mm}

\printlicense

\printsocials


\end{document}
