\documentclass{../cssheet}

%--------------------------------------------------------------------------------------------------------------
% Basic meta data
%--------------------------------------------------------------------------------------------------------------

\title{Konstruieren Üben!}
\author{Prof. Dr. Christian Spannagel}
\date{\today}
\hypersetup{%
    pdfauthor={\theauthor},%
    pdftitle={\thetitle},%
    pdfsubject={Aufgabenblatt Geometrie},%
    pdfkeywords={geometrie}
}


%--------------------------------------------------------------------------------------------------------------
% document
%--------------------------------------------------------------------------------------------------------------

\begin{document}
\printtitle

\textbf{Aufgabe 1 (Drei Büsche):}  In einem Garten stehen drei Büsche wild herum, die nicht mehr verpflanzt werden können. Es werden neue Büsche gekauft und derart eingepflanzt, dass alle Büsche in einem großen Kreis stehen. Wie findet man die Positionen der Büsche?

\textbf{Aufgabe 2 (Quadratur des Rechtecks):}  Konstruiert zu einem gegebenen Rechteck ein flächengleiches Quadrat.

\textbf{Aufgabe 3 (Trapezkonstruktion):}  Konstruiert ein Trapez $ABCD$ mit:\\ $DC || AB$, $l(a)=9~cm$, $l(b)=4,5~cm$, $l(c)=4~cm$, $l(d)=7,5~cm$, 

\textbf{Aufgabe 4 (Bierdeckelgeometrie):}  Ihr wollt einen Bierdeckel auf einem Finger balancieren. Findet den Mittelpunkt des Bierdeckels mit möglichst wenigen Konstruktionsschritten. 

\vspace*{5cm}
Die Aufgaben 1 bis 4 orientieren sich an Aufgaben aus: Benölken, R., Gorski, H.-J. \& Müller-Philipp. S. (2018). \emph{Leitfaden Geometrie} (7., überarb. u. erw. Aufl.; S. 252f.). Berlin, Heidelberg: Springer.


\vspace*{10mm}

\printlicense

\printsocials
\end{document}
