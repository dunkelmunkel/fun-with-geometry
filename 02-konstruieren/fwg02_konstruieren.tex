\documentclass{../cssheet}

%--------------------------------------------------------------------------------------------------------------
% Basic meta data
%--------------------------------------------------------------------------------------------------------------

\title{Konstruieren wie die Griechen}
\author{Prof. Dr. Christian Spannagel}
\date{\today}
\hypersetup{%
    pdfauthor={\theauthor},%
    pdftitle={\thetitle},%
    pdfsubject={Aufgabenblatt Geometrie},%
    pdfkeywords={geometrie}
}


%--------------------------------------------------------------------------------------------------------------
% document
%--------------------------------------------------------------------------------------------------------------

\begin{document}
\printtitle

\textbf{Vorbemerkung:} Die alten Griechen hatten nur zwei Werkzeuge zur Verfügung, mit denen sie geometrische Konstruktionen angefertigt haben: einen Stock oder Stab (ohne Markierungen) als Lineal und eine Schnur als Zirkel. Diese Werkzeuge nennt man auch \emph{euklidische Werkzeuge}. Heute konstruieren wir wie die alten Griechen und haben daher auch nur diese Werkzeuge zur Verfügung.

\textbf{Aufgabe 1 (Konstruiere wie die Griechen):}  
\begin{enumerate}[a)]
\item Konstruiere ein gleichseitiges Dreieck.
\item Konstruiere die Mittelsenkrechte einer Strecke.
\item Konstruiere eine Winkelhalbierende.
\item Konstruiere eine Senkrechte zu einer Geraden $g$ durch einen Punkt $P$.
\item Konstruiere eine Parallele zur einer Geraden $g$ durch einen Punkt $P$.
\item Konstruiere ein Quadrat.
\item Verlängere eine Strecke um den Faktor $\sqrt{2}$.
\item Teile eine Strecke per Konstruktion in drei gleich große Teilstrecken.
\item Teile eine Strecke per Konstruktion in fünf gleich große Teilstrecken.
\item Konstruiere eine Strecke der Länge $\sqrt{3}$, $\sqrt{6}$, $\sqrt{8}$, \ldots
\item Konstruiere eine Ellipse. (Achtung: Hier verwenden wir die Schnur mal ausnahmsweise nicht als Zirkel!)
\end{enumerate}


\vspace*{10mm}

\printlicense

\printsocials
\end{document}
