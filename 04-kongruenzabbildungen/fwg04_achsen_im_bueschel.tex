\documentclass{../cssheet}

%--------------------------------------------------------------------------------------------------------------
% Basic meta data
%--------------------------------------------------------------------------------------------------------------

\title{Achsen im Büschel}
\author{Prof. Dr. Christian Spannagel}
\date{\today}
\hypersetup{%
    pdfauthor={\theauthor},%
    pdftitle={\thetitle},%
    pdfsubject={Aufgabenblatt Geometrie},%
    pdfkeywords={geometrie}
}


%--------------------------------------------------------------------------------------------------------------
% document
%--------------------------------------------------------------------------------------------------------------

\begin{document}
\printtitle

\textbf{Aufgabe 1 (Parallele Achsen):} 
\begin{enumerate}[a)]
\item Beweist, dass das Produkt $a\circ b\circ c$ von drei Achsenspiegelungen an zueinander parallelen Achsen $a$, $b$ und $c$ wieder eine Achsenspiegelung $d$ ist. Wie kann man die Achse $d$ ermitteln? Verwendet gerne Geogebra!
\item Was ist das Produkt von 4, 6, 8, 10, \ldots Achsenspiegelungen an parallelen Achsen?
\item Was ist das Produkt von 3, 5, 7, 9, \ldots Achsenspiegelungen an parallelen Achsen?
\end{enumerate}

\textbf{Aufgabe 2 (Kopunktale Achsen):} 
\begin{enumerate}[a)]
\item Beweist, dass das Produkt $a\circ b\circ c$ von drei Achsenspiegelungen an kopunktalen (sich in einem Punkt schneidenden) Achsen $a$, $b$ und $c$ wieder eine Achsenspiegelung $d$ ist. Wie kann man die Achse $d$ ermitteln? Verwendet gerne Geogebra!
\item Was ist das Produkt von 4, 6, 8, 10, \ldots Achsenspiegelungen an kopunktalen Achsen?
\item Was ist das Produkt von 3, 5, 7, 9, \ldots Achsenspiegelungen an kopunktalen Achsen?
\end{enumerate}


Die Aufgaben orientieren sich an: Krauter, S. \& Bescherer, C. (2013). \emph{Erlebnis Elementargeometrie} (2. Aufl.). Berlin, Heidelberg: Springer. S.~30.

\vspace*{10mm}

\printlicense

\printsocials

\end{document}
