\documentclass[a4paper, 12pt]{mksheetphhd}


\title{Bandornamente}
\author{Marvin Ködding}
\DeclareMathOperator{\MOD}{\:MOD\:}
\DeclareMathOperator{\ggT}{ggT}
\usepackage{tikz}

%\sisetup{locale = DE}  

\begin{document}
	
	\printtitle
	
	\vspace*{0.2cm}
	
	\begin{aufgabe}
		Ein Bandornament ist ein unendlich ausgedehntes, eindimensionales Muster, das aus Abbildung eines Grundelements durch beliebig viele Kongruenzabbildungen entsteht. Es gibt insgesamt $7$ verschiedene Typen von Bandornamenten, die man durch die darin enthaltenen Kongruenzabbildungen kategorisieren kann. Untersucht, welche Kongruenzabbildungen in den verschiedenen Bandornamenten vorkommen:
	\end{aufgabe}
	
	\begin{center}
		\begin{tikzpicture}
			\draw[ultra thick] (-.2,1.2) -- (14.7, 1.2);
			\draw[ultra thick] (-.2,-.2) -- (14.7, -.2);
			\foreach \x in {0,1.5,...,13.5}
			{\draw[thick] (\x, 0.5) -- ++(.9, .5) -- ++(0, -1);}
		\end{tikzpicture}
		
		\begin{tikzpicture}
			\draw[ultra thick] (-.2,1.2) -- (14.7, 1.2);
			\draw[ultra thick] (-.2,-.2) -- (14.7, -.2);
			\foreach \x in {0,2.5,...,13}
			{
				\draw[thick] (\x, 0.5) -- ++(.8, .5) -- ++(0, -1);
				\draw[thick] (\x + 1.2, 1) -- ++(0, -1) -- ++(.8, .5);
			}
		\end{tikzpicture}
		
		\begin{tikzpicture}
			\draw[ultra thick] (-.2,1.2) -- (14.7, 1.2);
			\draw[ultra thick] (-.2,-.2) -- (14.7, -.2);
			\foreach \x in {0,2.5,...,13}
			{
				\draw[thick] (\x, 0.5) -- ++(.8, .5) -- ++(0, -1);
				\draw[thick] (\x + 1.2, 0) -- ++(0, 1) -- ++(.8, -.5);
			}
		\end{tikzpicture}
		
		\begin{tikzpicture}
			\draw[ultra thick] (-.2,1.2) -- (14.7, 1.2);
			\draw[ultra thick] (-.2,-.2) -- (14.7, -.2);
			\foreach \x in {0,2.5,...,13}
			{
				\draw[thick] (\x, 0.5) -- ++(.8, .5) -- ++(0, -1);
				\draw[thick] (\x + 1.2, .5) -- ++(0.8, -.5) -- ++(0, 1);
			}
		\end{tikzpicture}
		
		\begin{tikzpicture}
			\draw[ultra thick] (-.2,1.2) -- (14.7, 1.2);
			\draw[ultra thick] (-.2,-.2) -- (14.7, -.2);
			\foreach \x in {0,1.53,...,14}
			{
				\draw[thick] (\x, .9) -- ++(.8, 0) -- ++(0, -.8) -- ++(-.8, 0);
			}
		\end{tikzpicture}
		
		\begin{tikzpicture}
			\draw[ultra thick] (-.2,1.2) -- (14.7, 1.2);
			\draw[ultra thick] (-.2,-.2) -- (14.7, -.2);
			\foreach \x in {0,5,...,13}
			{
				\draw[thick] (\x, 0.5) -- ++(.8, .5) -- ++(0, -1);
				\draw[thick] (\x + 1.2, 0) -- ++(0, 1) -- ++(.8, -.5);
				\draw[thick] (\x + 2.4, 0.5) -- ++(.8, -.5) -- ++(0, 1);
				\draw[thick] (\x + 3.6, 1) -- ++(0, -1) -- ++(.8, .5);
			}
		\end{tikzpicture}
		
		\begin{tikzpicture}
			\draw[ultra thick] (-.2,1.2) -- (14.7, 1.2);
			\draw[ultra thick] (-.2,-.2) -- (14.7, -.2);
			\foreach \x in {0,2.5,...,14}
			{
				\draw[thick] (\x, .9) -- ++(.8, 0) -- ++(0, -.8) -- ++(-.8, 0);
				\draw[thick] (\x + 2, .9) -- ++(-.8, 0) -- ++(0, -.8) -- ++(.8, 0);
			}
		\end{tikzpicture}
		
	\end{center}
	
	
	\printlicense
	
	%\printsocials
	
\end{document}