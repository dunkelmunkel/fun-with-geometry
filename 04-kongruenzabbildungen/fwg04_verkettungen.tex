\documentclass{cssheet}

%--------------------------------------------------------------------------------------------------------------
% Basic meta data
%--------------------------------------------------------------------------------------------------------------

\title{Verkettungen}
\author{Prof. Dr. Christian Spannagel}
\date{\today}
\hypersetup{%
    pdfauthor={\theauthor},%
    pdftitle={\thetitle},%
    pdfsubject={Aufgabenblatt Geometrie},%
    pdfkeywords={geometrie}
}


%--------------------------------------------------------------------------------------------------------------
% document
%--------------------------------------------------------------------------------------------------------------

\begin{document}
\printtitle

\begin{aufgabe}[Verkettung von Abbildungen]

Begründet anhand von geeigneten Zeichnungen: 
\begin{enumerate}[a)]
\item Die Verkettung zweier Drehungen $D_{1}(P;\alpha)$ und $D_{2}(Q;\beta)$ ergibt wieder eine Drehung $D_{3}(R;\gamma=\alpha+\beta)$, falls die Summe der Drehwinkel kein Vielfaches von $360^{\circ}$ ist, und eine Verschiebung, falls die Summe der Drehwinkel ein Vielfaches von $360^{\circ}$ ist. Wie bestimmt man den neuen Drehpunkt bzw. Verschiebungsvektor?
\item Die Verkettung einer Drehung um den Winkel $\alpha$ mit einer Verschiebung ergibt eine Drehung um den Winkel $\alpha$. Wie bestimmt man den neuen Drehpunkt?
\item Die Verkettung einer Schubspiegelung mit einer Drehung ist eine Schubspiegelung.
\item Die Verkettung einer Schubspiegelung mit einer Verschiebung ist eine Schubspiegelung.
\item Was ergibt die Verkettung zweier Schubspiegelungen?

\end{enumerate}
\end{aufgabe}

\begin{aufgabe}[Dilatationen]

Punktspiegelungen und Verschiebungen bilden innerhalb der Kongruenzabbildungen die sogenannten Dilatationen.

Begründet anhand von geeigneten Zeichnungen: 
\begin{enumerate}[a)]
	\item Die Verkettung zweier Punktspiegelungen ergibt eine Verschiebung. Wie ergibt sich der Verschiebungsvektor?
	\item Die Verkettung von drei Punktspiegelungen ist wieder eine Punktspiegelung. Zentrum ist der die drei Punkte (in der gegebenen Reihenfolge) ergänzende vierte Parallelogrammpunkt. Was erhält man bei der Umkehrung der Reihenfolge?
	\item Die Verkettung einer ungeraden Anzahl von Punktspiegelungen ist wieder eine Punktspiegelung. Die Verkettung einer geraden Anzahl von Punktspiegelungen ist eine Verschiebung.
\end{enumerate}
\end{aufgabe}

Die Aufgaben stammen aus: Krauter, S. \& Bescherer, C. (2013). \emph{Erlebnis Elementargeometrie} (2. Aufl.). Berlin, Heidelberg: Springer. S.~36.

\vspace*{10mm}

\printlicense

\printsocials

\end{document}
