\documentclass{cssheet}

%--------------------------------------------------------------------------------------------------------------
% Basic meta data
%--------------------------------------------------------------------------------------------------------------

\title{Strahlensätze}
\author{Prof. Dr. Christian Spannagel}
\date{\today}
\hypersetup{%
	pdfauthor={\theauthor},%
	pdftitle={\thetitle},%
	pdfsubject={Aufgabenblatt Geometrie},%
	pdfkeywords={geometrie}
}


%--------------------------------------------------------------------------------------------------------------
% document
%--------------------------------------------------------------------------------------------------------------

\begin{document}
\printtitle

\begin{aufgabe} 
	Beweise den ersten Strahlensatz: Werden zwei sich schneidende Geraden von zwei Parallelen geschnitten, so verhalten sich die Abschnitte auf der einen Geraden wie die entsprechenden Abschnitte auf der anderen Geraden.
\end{aufgabe}

\begin{aufgabe} 
	Beweise den zweiten Strahlensatz: Werden zwei sich schneidende Geraden von zwei Parallelen geschnitten, so verhalten sich die vom Schnittpunkt aus gemessenen Abschnitte auf einer der Geraden wie die entsprechenden Abschnitte auf den Parallelen.
\end{aufgabe}

\begin{aufgabe} 
	Formuliere die Umkehrung des ersten Strahlensatzes und beweise sie.
\end{aufgabe}

\begin{aufgabe} 
	Formuliere die Umkehrung des zweiten Strahlensatzes und zeige durch ein Gegenbeispiel, dass sie \emph{nicht} gilt.
\end{aufgabe}

\begin{aufgabe} 
	Welche Verhältnisse gelten aufgrund der Strahlensätze in der untenstehenden Figur ($AC$ parallel zu $BD$)? Wie kann man die Richtigkeit begründen, wenn die Parallelen auf verschiedenen Seiten von $S$ liegen? (Hinweis: Punktspiegelung)
\end{aufgabe}

\vspace*{10mm}

Aufgaben orientiert an: Krauter, S. \& Bescherer, C. (2013). \emph{Erlebnis Elementargeometrie (2. Aufl.)}. Berlin, Heidelberg: Springer. S. 151–154

\printlicense

\printsocials

\end{document}
